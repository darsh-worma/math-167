\documentclass[letterpaper,11pt]{article}
\usepackage{graphicx}
\usepackage{abstract}
\usepackage{xpatch}
\renewcommand{\abstractname}{}
\renewcommand{\absnamepos}{empty} 
\usepackage[colorlinks,urlcolor=blue, linkcolor=black]{hyperref}
\usepackage[total={6in, 8in}]{geometry}
\usepackage[skip=0.6em plus 0.2em minus 0.1em]{parskip}
\usepackage{fancyhdr}
\usepackage{amsmath, amsfonts, amsthm, amssymb}
\usepackage{hyperref}
\usepackage{thmtools} 
\usepackage{mathtools}
\usepackage{tikz} 
\usetikzlibrary{shapes.geometric}

%----- NEW COMMANDS -----%
\newcommand{\N}[0]{\mathbb{N}}
\newcommand{\R}[0]{\mathbb{R}}
\newcommand{\Z}[0]{\mathbb{Z}}
\newcommand{\C}[0]{\mathbb{C}}
\newcommand{\Q}[0]{\mathbb{Q}}
\newcommand{\im}[0]{\text{Im }}
\renewcommand{\ker}[0]{\text{Ker }}
\renewcommand{\H}[0]{\mathbb{H}}
\renewcommand{\mod}[0]{\text{ mod }}

%----- THEOREM SET-UP -----%
% Break line after the head + give generous space below
\newtheoremstyle{thm-break}%
  {6pt}%   space above
  {12pt plus 3pt minus 2pt}% space below
  {\itshape}% body font
  {}% indent
  {\bfseries}% head font
  {.}% punctuation after head
  {\newline}% space after head (ignored—we’ll break in the head spec)
  {%
    \thmname{#1}\thmnumber{ #2}\thmnote{ (#3)}%
    \par\nobreak\vspace{0.25\baselineskip}% ← guaranteed line break + a touch of space
  }

\newtheoremstyle{def-inline}%
  {6pt}%   space above
  {12pt plus 3pt minus 2pt}% space below (keeps the post-env gap)
  {\normalfont}% body font
  {}% indent
  {\bfseries}% head font
  {.}% punctuation after head
  {0.5em}% space after head (inline, not newline)
  {}% head spec

\newtheoremstyle{indented-style}% name of the style
  {3pt}% space above
  {3pt}% space below
  {\itshape\addtolength\leftskip{2em}}% body font (indents the whole block)
  {}% indent amount for first line (we don't need it)
  {\bfseries}% theorem head font
  {.}% punctuation after theorem head
  {.5em}% space after theorem head
  {}% theorem head spec

% % Use styles
% \theoremstyle{thm-break}
% \newtheorem{theorem}{Theorem}[section]
% \newtheorem{lemma}[theorem]{Lemma}
% \newtheorem{example}[theorem]{Example}

% % Corollaries numbered under the most recent theorem
% \newtheorem{corollary}{Corollary}[theorem]
% \makeatletter
% \renewcommand{\thecorollary}{\thetheorem.\arabic{corollary}}
% \makeatother

% \xapptocmd{\proof}{\mbox{}\par\nobreak}{}{}

% % Unnumbered envs (but with the def-break spacing + newline)
% \theoremstyle{def-inline}
% \newtheorem*{definition}{Definition}
% \newtheorem*{remark}{Remark}

% Use your existing styles
\theoremstyle{thm-break}
\declaretheorem[name=Theorem,numberwithin=section]{theorem}
\declaretheorem[name=Lemma,sibling=theorem]{lemma}
\declaretheorem[name=Proposition,sibling=theorem]{proposition}

% If you want corollaries to share the theorem counter (Theorem-style numbering)
% \declaretheorem[name=Corollary,numberlike=theorem]{corollary}
\newtheorem{corollary}{Corollary}[theorem]
\makeatletter
\renewcommand{\thecorollary}{\thetheorem.\arabic{corollary}}
\makeatother

% Unnumbered environments still fine; they won't appear in the list by default
\theoremstyle{def-inline}
\declaretheorem[name=Definition,numbered=no]{definition}
\declaretheorem[name=Remark,numbered=no]{remark}
\declaretheorem[name=Example,numberwithin=section]{example}
\declaretheorem[name=Intuition,style=indented-style,numbered=no]{intuition}

%----- HEADER -----%
\pagestyle{fancy}
\fancyhead[L]{\textbf{Darsh Verma} (Winter 2026)}
\fancyhead[R]{\rightmark}

%----- TITLE INFO -----%
\title{\LARGE\textbf{Math 167: Game Theory}\\
{\Large UCLA}}
\author{\large Darsh Verma}
\date{\large Winter 2026}

%--------------- DOCUMENT BEGINS ----------------%

\begin{document}

\maketitle
\begin{abstract}
    Hello and welcome! As the title suggests, these are my lecture notes on Game Theory. Our professor is \textbf{Sylvester Zhang}. The textbook that we are using is \textbf{Game Theory, Alive by Anna R. Karlin and Yuval Peres}.
    
    The goal of these lecture notes is to write \textbf{understandable} math. Some dude said, "If you can't explain it to a six year old, then you don't understand it yourself". The hope is that anyone coming across these notes (like you!) will be able to at least take away the gist of these concepts. Email me at darsh [at] ucla [dot] edu if you find any errors!

    Huge shoutout to \url{https://zitong.me/notes/rings-notes.pdf} who inspired me to attend class and lock in.
\end{abstract}

\tableofcontents
% Only list the theorem-like things you care about
\newpage
\listoftheorems[
  title={List of Definitions},
  ignoreall,               % start from empty
  show={definition} % choose which to include
]
\listoftheorems[
  title={List of Theorems},
  ignoreall,               % start from empty
  show={theorem} % choose which to include
]

\newpage
\section{Lecture 1: Jan 5}
\subsection{Introduction}
Today, the professor arrived 20 minutes late so we just did a brief intro on things we're gonna do in the class like combinatorial games, two-person zero sum games, general sum games, Nash equilibria, fixed-point theorem, and evolutionary models.

Apparently, the fixed point theorem is used to prove something related to the Nash equilibria. Interestingly enough, Zitong sent me \href{https://www.instagram.com/p/DNoxJ-khwvu/}{this reel} a few days ago where I first learned about the fixed point theorem.

We ended lecture by playing the classic $4 \times 5$ version of \href{https://en.wikipedia.org/wiki/Chomp}{Chomp}! It seems to me that the first player always has a winning strategy but I need to formalize why this is true.

\end{document}
