\documentclass[letterpaper,11pt]{article}
\usepackage{graphicx}
\usepackage{abstract}
\usepackage{xpatch}
\renewcommand{\abstractname}{}
\renewcommand{\absnamepos}{empty} 
\usepackage[colorlinks,urlcolor=blue, linkcolor=black]{hyperref}
\usepackage[total={6in, 8in}]{geometry}
\usepackage[skip=0.6em plus 0.2em minus 0.1em]{parskip}
\usepackage{fancyhdr}
\usepackage{amsmath, amsfonts, amsthm, amssymb}
\usepackage{hyperref}
\usepackage{thmtools} 
\usepackage{mathtools}
\usepackage{tikz} 
\usetikzlibrary{shapes.geometric}
\usetikzlibrary{arrows.meta}
\usetikzlibrary{decorations.pathreplacing}

%----- NEW COMMANDS -----%
\newcommand{\N}[0]{\mathbb{N}}
\newcommand{\R}[0]{\mathbb{R}}
\newcommand{\Z}[0]{\mathbb{Z}}
\newcommand{\C}[0]{\mathbb{C}}
\newcommand{\Q}[0]{\mathbb{Q}}
\newcommand{\im}[0]{\text{Im }}
\renewcommand{\ker}[0]{\text{Ker }}
\renewcommand{\H}[0]{\mathbb{H}}
\renewcommand{\mod}[0]{\text{ mod }}

%----- THEOREM SET-UP -----%
% Break line after the head + give generous space below
\newtheoremstyle{thm-break}%
  {6pt}%   space above
  {12pt plus 3pt minus 2pt}% space below
  {\itshape}% body font
  {}% indent
  {\bfseries}% head font
  {.}% punctuation after head
  {\newline}% space after head (ignored—we’ll break in the head spec)
  {%
    \thmname{#1}\thmnumber{ #2}\thmnote{ (#3)}%
    \par\nobreak\vspace{0.25\baselineskip}% ← guaranteed line break + a touch of space
  }

\newtheoremstyle{def-inline}%
  {6pt}%   space above
  {12pt plus 3pt minus 2pt}% space below (keeps the post-env gap)
  {\normalfont}% body font
  {}% indent
  {\bfseries}% head font
  {.}% punctuation after head
  {0.5em}% space after head (inline, not newline)
  {}% head spec

\newtheoremstyle{indented-style}% name of the style
  {3pt}% space above
  {3pt}% space below
  {\itshape\addtolength\leftskip{2em}}% body font (indents the whole block)
  {}% indent amount for first line (we don't need it)
  {\bfseries}% theorem head font
  {.}% punctuation after theorem head
  {.5em}% space after theorem head
  {}% theorem head spec

% % Use styles
% \theoremstyle{thm-break}
% \newtheorem{theorem}{Theorem}[section]
% \newtheorem{lemma}[theorem]{Lemma}
% \newtheorem{example}[theorem]{Example}

% % Corollaries numbered under the most recent theorem
% \newtheorem{corollary}{Corollary}[theorem]
% \makeatletter
% \renewcommand{\thecorollary}{\thetheorem.\arabic{corollary}}
% \makeatother

% \xapptocmd{\proof}{\mbox{}\par\nobreak}{}{}

% % Unnumbered envs (but with the def-break spacing + newline)
% \theoremstyle{def-inline}
% \newtheorem*{definition}{Definition}
% \newtheorem*{remark}{Remark}

% Use your existing styles
\theoremstyle{thm-break}
\declaretheorem[name=Theorem,numberwithin=section]{theorem}
\declaretheorem[name=Lemma,sibling=theorem]{lemma}
\declaretheorem[name=Proposition,sibling=theorem]{proposition}

% If you want corollaries to share the theorem counter (Theorem-style numbering)
% \declaretheorem[name=Corollary,numberlike=theorem]{corollary}
\newtheorem{corollary}{Corollary}[theorem]
\makeatletter
\renewcommand{\thecorollary}{\thetheorem.\arabic{corollary}}
\makeatother

% Unnumbered environments still fine; they won't appear in the list by default
\theoremstyle{def-inline}
\declaretheorem[name=Definition,numbered=no]{definition}
\declaretheorem[name=Remark,numbered=no]{remark}
\declaretheorem[name=Example,numberwithin=section]{example}
\declaretheorem[name=Intuition,style=indented-style,numbered=no]{intuition}

%----- HEADER -----%
\pagestyle{fancy}
\fancyhead[L]{\textbf{Darsh Verma} (Winter 2026)}
\fancyhead[R]{\rightmark}

%----- TITLE INFO -----%
\title{\LARGE\textbf{Math 167: Game Theory}\\
{\Large UCLA}}
\author{\large Darsh Verma}
\date{\large Winter 2026}

%--------------- DOCUMENT BEGINS ----------------%

\begin{document}

\maketitle
\begin{abstract}
    Hello and welcome! As the title suggests, these are my lecture notes on Game Theory. Our professor is \textbf{Sylvester Zhang}. The textbook that we are using is \textbf{Game Theory, Alive by Anna R. Karlin and Yuval Peres}.
    
    The goal of these lecture notes is to write \textbf{understandable} math. Some dude said, "If you can't explain it to a six year old, then you don't understand it yourself". The hope is that anyone coming across these notes (like you!) will be able to at least take away the gist of these concepts. Email me at darsh [at] ucla [dot] edu if you find any errors!

    Huge shoutout to \url{https://zitong.me/notes/rings-notes.pdf} who inspired me to attend class and lock in.
\end{abstract}

\tableofcontents
% Only list the theorem-like things you care about
\newpage
\listoftheorems[
  title={List of Definitions},
  ignoreall,               % start from empty
  show={definition} % choose which to include
]
\listoftheorems[
  title={List of Theorems},
  ignoreall,               % start from empty
  show={theorem} % choose which to include
]

\newpage
\section{Lecture 1: Jan 5}
\subsection{Introduction}
Today, the professor arrived 20 minutes late so we just did a brief intro on things we're gonna do in the class like combinatorial games, two-person zero sum games, general sum games, Nash equilibria, fixed-point theorem, and evolutionary models.

Apparently, the fixed point theorem is used to prove something related to the Nash equilibria. Interestingly enough, Zitong sent me \href{https://www.instagram.com/p/DNoxJ-khwvu/}{this reel} a few days ago where I first learned about the fixed point theorem.

We ended lecture by playing the classic $4 \times 5$ version of \href{https://en.wikipedia.org/wiki/Chomp}{Chomp}! It seems to me that the first player always has a winning strategy but I need to formalize why this is true.
\newpage
\section{Jan 7 : Lecture 2}
\subsection{Impartial Combinatorial games}
\begin{itemize}
  \item two-player games, alternate turns
  \item perfect information
  \item no randomess
  \item both players have the same set of moves
  \item player who takes the last move wins
  \item win or loss outcome
\end{itemize}
\begin{example}
  There are $n$ chips on a table. There are two players, Larry and Rick. A valid move is to take 1, 2, or 3 chips from the pile. Assume that Larry goes first and the player who takes the last chip wins.
  \\
  We proceed with backward induction. Let's define the following states: \\
  $N$ : the next player to take a move wins. \\
  $P$ : the previous player that took a move won. \\
  These are conventions because you can have multiple players instead of just two. We could have also just called them you and your opponent.


  Backward induction just means that we start analyzing states by having 0 chips on the table. That falls under state $P$. This implies that if there are 1, 2, or 3 chips on the table, those fall under state $N$ because the next player who moves can just take 1, 2, or 3 chips and win the game. Similarly, we can extend this logic that 4 chips falls under state $P$, and so on. \\
  It turns out that the Larry has a 75\% chance of winning, and Rick has a 25\% chance of winning simply because of who went first.

  
\end{example}
\begin{example}[Chomp]
  Given a $2 \times 3$ chocolate bar,  here are the possible states it can go to
\end{example}
\begin{center}
\begin{tikzpicture}[
  square/.style={minimum size=0.6cm, draw, thick},
  chocolate/.style={square, fill=brown!60},
  poison/.style={square, fill=red!40},
  eaten/.style={square, fill=white, draw=gray!50, dashed},
  arrow/.style={-{Stealth[length=3mm]}, thick, gray}
]

% Initial 2x3 bar (center top)
\begin{scope}[shift={(0,3)}]
  \node[poison] at (0,0) {};
  \node[chocolate] at (0.6,0) {};
  \node[chocolate] at (1.2,0) {};
  \node[chocolate] at (0,0.6) {};
  \node[chocolate] at (0.6,0.6) {};
  \node[chocolate] at (1.2,0.6) {};
 % \node at (0.6,-0.5) {\small Initial};
\end{scope}

% State 1: Remove top-right (1,2) -> 2x2 + 1
\begin{scope}[shift={(-5,0)}]
  \node[poison] at (0,0) {};
  \node[chocolate] at (0.6,0) {};
  \node[chocolate] at (1.2,0) {};
  \node[chocolate] at (0,0.6) {};
  \node[chocolate] at (0.6,0.6) {};
  \node[eaten] at (1.2,0.6) {};
  \node at (0.6,-0.5) {\small 1};
\end{scope}

% State 2: Remove top-middle and top-right -> 1x3 + 1x1
\begin{scope}[shift={(-2.5,0)}]
  \node[poison] at (0,0) {};
  \node[chocolate] at (0.6,0) {};
  \node[chocolate] at (1.2,0) {};
  \node[chocolate] at (0,0.6) {};
  \node[eaten] at (0.6,0.6) {};
  \node[eaten] at (1.2,0.6) {};
  \node at (0.6,-0.5) {\small 2};
\end{scope}

% State 3: Remove entire top row -> 1x3
\begin{scope}[shift={(0,0)}]
  \node[poison] at (0,0) {};
  \node[chocolate] at (0.6,0) {};
  \node[chocolate] at (1.2,0) {};
  \node[eaten] at (0,0.6) {};
  \node[eaten] at (0.6,0.6) {};
  \node[eaten] at (1.2,0.6) {};
  \node at (0.6,-0.5) {\small 3};
\end{scope}

% State 4: Remove right column -> 2x2
\begin{scope}[shift={(2.5,0)}]
  \node[poison] at (0,0) {};
  \node[chocolate] at (0.6,0) {};
  \node[eaten] at (1.2,0) {};
  \node[chocolate] at (0,0.6) {};
  \node[chocolate] at (0.6,0.6) {};
  \node[eaten] at (1.2,0.6) {};
  \node at (0.6,-0.5) {\small 4};
\end{scope}

% State 5: Remove two right columns -> 2x1
\begin{scope}[shift={(5,0)}]
  \node[poison] at (0,0) {};
  \node[eaten] at (0.6,0) {};
  \node[eaten] at (1.2,0) {};
  \node[chocolate] at (0,0.6) {};
  \node[eaten] at (0.6,0.6) {};
  \node[eaten] at (1.2,0.6) {};
  \node at (0.6,-0.5) {\small 5};
\end{scope}

% Arrows from initial state to all 5 states
\draw[arrow] (-0.3,2.5) -- (-4.4,1.1);
\draw[arrow] (0.3,2.5) -- (-1.9,1.1);
\draw[arrow] (0.6,2.5) -- (0.6,1.1);
\draw[arrow] (0.9,2.5) -- (3.1,1.1);
\draw[arrow] (1.5,2.5) -- (5.6,1.1);

\end{tikzpicture}
\end{center}
We ended class with a quiz on induction.

\newpage
\section{Lecture 3 : Jan 9}
I'm not sure what these initial definitions are but I'll understand later.
\begin{definition}
  $P_0 = P_1 = \{\text{terminal position}\}$ \\
  $N_{n+1} = \{x \mid \text{there exists a move } x \rightarrow y, y \in P_n\}$ \\
  $P_{n+1} = \{x \mid \text{there exists a move } x \rightarrow y, y \in N_n\}$
\end{definition}

\begin{definition}[Progressively Bounded]
  A game is called progressive bounded if for every position $x$, there exists an upper bound $B(x) \in \Z_{\geq 0}$ on the number of moves until the game stops.
  
\end{definition}

Now, we will talk about potential winning strategies for the Game Chomp.

\begin{proposition}[Chomp Winners be like]
  For any rectangular $n \times m$ Chomp game, the player that goes first wins!
\end{proposition}
\begin{proof}
  If we have a square board, start off by removing everything from the square up and to the right of the poisoned square. Then, just mimic your opponent.
  \begin{center}
  \begin{tikzpicture}[
    square/.style={minimum size=0.6cm, draw, thick},
    chocolate/.style={square, fill=brown!60},
    poison/.style={square, fill=red!40},
    eaten/.style={square, fill=white, draw=gray!50, dashed},
    arrow/.style={-{Stealth[length=3mm]}, thick, gray}
  ]
  
  % Complete 3x3 grid (left side)
  \begin{scope}[shift={(0,0)}]
    \node[poison] at (0,0) {};
    \node[chocolate] at (0.6,0) {};
    \node[chocolate] at (1.2,0) {};
    \node[chocolate] at (0,0.6) {};
    \node[chocolate] at (0.6,0.6) {};
    \node[chocolate] at (1.2,0.6) {};
    \node[chocolate] at (0,1.2) {};
    \node[chocolate] at (0.6,1.2) {};
    \node[chocolate] at (1.2,1.2) {};
  \end{scope}
  
  % Arrow P1
  \draw[arrow] (2,0.6) -- (3.5,0.6) node[midway, above] {\small $P_1$};
  
  % L-shaped result after P1's move
  \begin{scope}[shift={(4.5,0)}]
    \node[poison] at (0,0) {};
    \node[chocolate] at (0.6,0) {};
    \node[chocolate] at (1.2,0) {};
    \node[chocolate] at (0,0.6) {};
    \node[eaten] at (0.6,0.6) {};
    \node[eaten] at (1.2,0.6) {};
    \node[chocolate] at (0,1.2) {};
    \node[eaten] at (0.6,1.2) {};
    \node[eaten] at (1.2,1.2) {};
  \end{scope}
  
  % Arrow P2
  \draw[arrow] (6.5,0.6) -- (8,0.6) node[midway, above] {\small $P_2$};
  
  % After P2 removes bottom right
  \begin{scope}[shift={(9,0)}]
    \node[poison] at (0,0) {};
    \node[chocolate] at (0.6,0) {};
    \node[eaten] at (1.2,0) {};
    \node[chocolate] at (0,0.6) {};
    \node[eaten] at (0.6,0.6) {};
    \node[eaten] at (1.2,0.6) {};
    \node[chocolate] at (0,1.2) {};
    \node[eaten] at (0.6,1.2) {};
    \node[eaten] at (1.2,1.2) {};
  \end{scope}
  
  % Arrow P1 mimic
  \draw[arrow] (10.5,0.6) -- (12,0.6) node[midway, above] {\small $P_1$};
  
  % After P1 mimics by removing top left
  \begin{scope}[shift={(13,0)}]
    \node[poison] at (0,0) {};
    \node[chocolate] at (0.6,0) {};
    \node[eaten] at (1.2,0) {};
    \node[chocolate] at (0,0.6) {};
    \node[eaten] at (0.6,0.6) {};
    \node[eaten] at (1.2,0.6) {};
    \node[eaten] at (0,1.2) {};
    \node[eaten] at (0.6,1.2) {};
    \node[eaten] at (1.2,1.2) {};
  \end{scope}
  
  \end{tikzpicture}
  \end{center}
  
  Now, we will prove the existence of a winning strategy for a rectangular board.
  Make a harmless move by removing the top right square. Now, that board is either a $P$ state or an $N$ state. \\
  Assume we are in an $N$ state. Whatever move Player 2 makes, Player 1 could have made that same move when they went first. So Player 1 wins. \\
  On the other hand, being in a $P$ state implies that Player 2 already lost. this is dumbass proof but wtv
\end{proof}
\begin{theorem}
  Let $x$ be a state of a progressively bounded impartial combinatorial game. Then, $ x \in N \cup P$ and $N \cap P = \emptyset$. 
\end{theorem}
\begin{proof}
  This theorem is basically saying that every state is either state $N$ or state $P$ but never both. And from any state, there always exists a winning strategy. Since we are in a progressively bounded game, we have $B(x)$. We are gonna induct on $n = B(x)$. \\
  The base case is trivial apparently. \\
  Our inductive hypothesis: if $B(x) = n$, then $x \in P_n \cup N_n$, and $P_n \cap N_n = \emptyset$. \\
  For our inductive step, we need to prove if $B(x) = n + 1$, then $x \in P_{n + 1} \cup N_{n + 1}$, and $P_{n + 1} \cap N_{n + 1} = \emptyset$. \\
  Case 1: for any move $x \rightarrow y$, $y \in N_n \implies x \in P_{n + 1}$ \\
  Case 2: there exists a move $x \rightarrow y$, $y \in P_n \implies x \in N_{n + 1}$
\end{proof}
\subsection{Nim}
  There are several piles of finitely many chips. A player can remove any number of chips from a single pile. Players alternate in turns. The player who takes the last chip wins. \\ \\
  To start off, let's think about a game with 2 piles with an arbitrary number of chips in them. If one pile has $a$ chips, and the other has $b$ chips this is equivalent to a Chomp game as follows:
  \begin{center}
  \begin{tikzpicture}[
    square/.style={minimum size=0.6cm, draw, thick},
    chocolate/.style={square, fill=brown!60},
    poison/.style={square, fill=red!40},
  ]
  
  % Bottom row (a chips, not counting poison)
  \node[poison] at (0,0) {};
  \node[chocolate] at (0.6,0) {};
  \node[chocolate] at (1.2,0) {};
  \node at (1.8,0) {$\cdots$};
  \node[chocolate] at (2.4,0) {};
  
  % Left column (b chips, not counting poison)
  \node[chocolate] at (0,0.6) {};
  \node[chocolate] at (0,1.2) {};
  \node at (0,1.8) {$\vdots$};
  \node[chocolate] at (0,2.4) {};
  
  % Brace for bottom row (a chips)
  \draw[decorate, decoration={brace, amplitude=5pt, mirror}] (0.3,-0.5) -- (2.7,-0.5) node[midway, below=5pt] {$a$};
  
  % Brace for left column (b chips)
  \draw[decorate, decoration={brace, amplitude=5pt}] (-0.5,0.3) -- (-0.5,2.7) node[midway, left=5pt] {$b$};
  
  \end{tikzpicture}
  \end{center}
  We ended class by thinking about the strategy if we had 3 piles.


  \newpage
  \section{Lecture 4: January 12}
  We recapped the game of Nim, and started talking about binary numbers for some reason. I'm trying to think about how to motivate thinking about binary numbers in this context.

  \subsection{A Bit of Nim Strategy}
  \begin{definition}[Nim Sum, $\oplus$]
    $$0 \oplus 0 = 1 \oplus 1 = 0$$
    $$1 \oplus 0 = 0 \oplus 1 = 1$$ 
  \end{definition}
  \begin{example}
    $ 3 \oplus 5 = 6$. Convert $3$ and $5$ to binary, add them without carrying anything over, and convert back to decimal.
  \end{example} 
  \begin{theorem}
    A state $(x_1, \dots, x_k)$ is a $P$-position if and only if $x_1 \oplus \dots \oplus x_k = 0$.
  \end{theorem}
  \begin{proof}
    We first claim that at any state $(x_1, \dots, x_k)$ with non-zero Nim sum, there exists a move to a zero Nim sum state. We can illustrate this with an example.

    We also claim that from a zero Nim sum, any move will change the Nim sum to non zero. 

    When $x_1 = \dots = x_k = 0$, this is a $P$ state. The Nim sum is $0 \oplus \dots 0 = 0$. 

    Claim 1: each $N$ position has a move to a $P$ position. \\
    Claim 2: All moves from a $P$ positiion are going to $N$ position.

    Complete this proof by induction.
  \end{proof}

  \subsection{Two Person Zero Sum Games}
  $P_1$ has a non empty set of strategies $S_1$. Similarly, $P_2$ has a non empty set of strategies $S_2$. \\ 
  A function $A : S_1 \times S_2 \rightarrow \R$ the payoff function for $P_1$.
  A function $A' : S_1 \times S_2 \rightarrow \R$ the payoff function for $P_2$. \\
  Then, $A'(s_1, s_2) = -A(s_1, s_2), \forall s_1 \in S_1, s_2 \in S_2$.

  \begin{center}
  \begin{tabular}{c|cccc}
    & $s_{21}$ & $s_{22}$ & $\cdots$ & $s_{2n}$ \\
    \hline
    $s_{11}$ & $a_{11}$ & $a_{12}$ & $\cdots$ & $a_{1n}$ \\
    $s_{12}$ & $a_{21}$ & $a_{22}$ & $\cdots$ & $a_{2n}$ \\
    $\vdots$ & $\vdots$ & $\vdots$ & $\ddots$ & $\vdots$ \\
    $s_{1m}$ & $a_{m1}$ & $a_{m2}$ & $\cdots$ & $a_{mn}$ \\
  \end{tabular}
  \end{center}
  Note that in these games both players choose their strategy simultaneously like rock paper scissors. In the combinatorial games, the strategy is dependent on what your opponent just played.

  Player 1 bets on $\displaystyle\min_{1 \leq j \leq n} a_{ij}$ and player 2 chooses 
  $$\max_{1 \leq i \leq m} \, \min_{1 \leq j \leq n} a_{ij}$$

  \begin{theorem}[Min Max Theorem]
    max min commute in the above thing.
  \end{theorem}

\newpage
\section{Lecture 5: January 14}
We started class with a quiz on $N$ and $P$ states. \\
\begin{definition}[Pure Strategy]
  Both players pick one strategy simultaneously
\end{definition}
\begin{definition}[Mixed Strategy]
  A vector $(p_1, p_2, \dots, p_m)$ such that $\sum_i p_i = 1$. Will a professional poker player always beat a random player?
\end{definition}
Player 1 bets on $\min_{1 \leq j \leq n} a_{ij}$ and chooses $\max_{1 \leq i \leq m} min_{1 \leq j \leq n} a_{ij}$. \\
Player 2 bets on $\min_{1 \leq j \leq n} a_{ij}$ and chooses $\max_{1 \leq i \leq m} min_{1 \leq j \leq n} a_{ij}$. \\

\begin{lemma}
  $$\max_i \min_j a_{ij} \leq \min_j \max_i a_{ij}$$
\end{lemma}
\begin{proof}
  Suppose $\max_i \min_j = a_{pi}$ and $\min_j \max_i a_{ij} = a_{rs}$.
  $a_{pq} \leq a_{ps}$ and $a_{ps} \leq a_{rs}$
  $a_{pq} \leq a_{rs}$
\end{proof}

\end{document}
